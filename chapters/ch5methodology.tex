\section{System Block Diagram}

\begin{figure}[h!]
    \centering
    \begin{tikzpicture}[node distance=2cm, auto, >=latex', thick]
        % Sensor nodes
        \node[draw, rectangle, rounded corners, minimum width=2cm, minimum height=1cm] (mpu) {MPU6050};
        \node[draw, rectangle, rounded corners, minimum width=2cm, minimum height=1cm, below left=1cm and 2cm of mpu] (flex) {Flex Sensors};
        \node[draw, rectangle, rounded corners, minimum width=2cm, minimum height=1cm, below right=1cm and 2cm of mpu] (tactile) {Tactile Switches};
        
        % Processing node
        \node[draw, rectangle, minimum width=3cm, minimum height=1cm, below=3cm of mpu] (micro) {Arduino Mega};
        
        % Calibration node
        \node[draw, rectangle, minimum width=2.5cm, minimum height=0.8cm, below=1.5cm of micro] (calibration) {Data Calibration};
        
        % Communication node
        \node[draw, rectangle, minimum width=3cm, minimum height=1cm, below=1.5cm of calibration] (comm) {Serial Communication};
        
        % Software nodes
        \node[draw, rectangle, minimum width=3cm, minimum height=1cm, below=1.5cm of comm] (pc) {PC Interface};
        \node[draw, rectangle, minimum width=3cm, minimum height=1cm, below=1.5cm of pc] (unreal) {Unreal Engine};
        
        % Output node
        \node[draw, rectangle, minimum width=3.5cm, minimum height=1cm, below=1.5cm of unreal] (virtual) {3D Hand Model};

        % Connections
        \draw[->] (mpu) -- (micro);
        \draw[->] (flex) -- (micro);
        \draw[->] (tactile) -- (micro);
        \draw[->] (micro) -- (calibration);
        \draw[->] (calibration) -- (comm);
        \draw[->] (comm) -- (pc);
        \draw[->] (pc) -- (unreal);
        \draw[->] (unreal) -- (virtual);

    \end{tikzpicture}
    \caption{Detailed Block Diagram of H.A.N.D}
    \label{fig:block_diagram}
\end{figure}

\section{Algorithm}

\begin{itemize}
    \item Configure MPU6050 for motion tracking
    \item Set up analog inputs for flex sensors
    \item Configure digital pins for tactile switches
    \item Initialize serial communication port
    \item Read acceleration and rotation data from MPU6050
    \item Sample voltage values from flex sensors
    \item Check state of tactile switches
    \item Apply noise filtering to sensor readings
    \item Implement calibration offsets to raw data
    \item Calculate hand orientation from accelerometer/gyroscope data
    \item Convert flex sensor readings to finger joint angles
    \item Prepare data packet with structured format
    \item Calculate and append checksum for error detection
    \item Transmit formatted data through serial interface
    \item Receive data packets on PC side
    \item Validate incoming data integrity
    \item Update virtual hand skeleton parameters
    \item Apply calculated movements to 3D hand model
\end{itemize}

\section{Flowchart}
\begin{figure}[h!]
    \centering
    \begin{tikzpicture}[node distance=1.5cm, auto, >=latex', thick,
                        block/.style={rectangle, draw, text width=4.5cm, text centered, rounded corners, minimum height=0.8cm, fill=blue!10},
                        start/.style={ellipse, draw, text width=3.5cm, text centered, minimum height=0.7cm, fill=green!20},
                        decision/.style={diamond, draw, text width=2.5cm, text centered, minimum height=0.9cm, fill=yellow!20},
                        line/.style={draw, -latex', thick}]
        
        % Nodes
        \node [start] (start) {Start};
        \node [block, below=of start] (init) {Initialize Hardware};
        \node [block, below=of init] (acquire) {Collect Sensor Data};
        \node [block, below=of acquire] (process) {Process Data};
        \node [block, below=of process] (transmit) {Transmit Data};
        \node [block, below=of transmit] (update) {Render Model};
        \node [decision, below=of update] (continue) {Continue?};
        \node [start, below=of continue] (end) {End};
        
        % Connections
        \path [line] (start) -- (init);
        \path [line] (init) -- (acquire);
        \path [line] (acquire) -- (process);
        \path [line] (process) -- (transmit);
        \path [line] (transmit) -- (update);
        \path [line] (update) -- (continue);
        \path [line] (continue) -- node[near start] {No} (end);
        \path [line] (continue.west) -- ++(-2,0) |- node[near start] {Yes} (acquire.west);
        
    \end{tikzpicture}
    \caption{Algorithm Flowchart for H.A.N.D}
    \label{fig:algorithm_flow}
\end{figure}



