\section{Background Theory}
The rapid advancement of Unmanned Aerial Vehicles (UAVs) has transformed applications in environmental monitoring, agriculture, and urban planning by providing cost-effective and high-resolution data collection capabilities [1], [2]. Fixed-wing UAVs, in particular, offer significant advantages for land use image classification due to their extended range, longer endurance, and ability to cover large areas efficiently, making them ideal for wide-areas surveillance and mapping tasks [2]. These vehicles can capture high-resolution aerial imagery, enabling precise analysis of land use patterns critical for precision agriculture, urban development, and disaster management [1], [4]. However, challenges such as limited payload capacity, regulatory constraints, and the need for advanced image processing to achieve high-quality outputs persist, necessitating innovative design and algorithmic solutions [1], [2].

A key component of this project is the integration of Enhanced Super-Resolution Generative Adversarial Network (ESRGAN), a state-of-the-art deep learning model for single image super-resolution (SISR) introduced by Wang et al. in 2018 [5]. ESRGAN employs a Generative Adversarial Network (GAN) framework, comprising a generator that creates high-resolution images from low-resolution inputs and a discriminator that ensures realism. Noted for its ability to reconstruct realistic textures and fine details, ESRGAN is ideal for enhancing images captured by UAV-mounted cameras, such as those using Raspberry Pi modules, typically upscaling images by a factor of 4x [5]. This capability is particularly valuable for land use classification, where high-resolution imagery is essential for identifying intricate land features.

This research project aims to design and develop a fixed-wing UAV tailored for land use image classification, leveraging lightweight materials, advanced sensors, and ESRGAN for enhanced image processing to improve data accuracy and operational efficiency. By addressing design considerations such as aerodynamic efficiency, flight stability, and integration of high-resolution imaging systems, this project seeks to create a UAV capable of collecting and processing aerial imagery for accurate land use analysis [2]. Drawing on prior research, including deep learning applications for UAV systems and their use in precision agriculture [1], [4], this project will optimize UAV performance for environmental monitoring tasks. The outcome is expected to contribute to UAV-based remote sensing, offering a scalable solution for land use analysis in diverse geographical contexts.

\section{Problem Statement}
Land use and land cover (LULC) classification is critical for effective environmental management, agricultural monitoring, urban planning, and disaster response. Traditionally, satellite imagery has been the primary source for such analysis; however, limitations such as low spatial resolution, infrequent data updates, atmospheric disturbances, and high operational costs restrict its applicability for high-precision, real-time monitoring [1]. In contrast, UAVs offer a flexible, cost-effective, and scalable solution for high-resolution remote sensing. Among UAV types, fixed-wing aircraft present superior advantages for surveying large geographical areas due to their extended flight duration, aerodynamic efficiency, and higher operational altitudes [2].

Despite their potential, existing UAV platforms used for LULC classification are often either commercially expensive, overly complex for academic research and field adaptation, or designed around rotary-wing systems that lack the range and endurance needed for broad-area coverage [3]. Moreover, there is a lack of integrated, low-cost fixed-wing UAV solutions that combine autonomous navigation, high-resolution image acquisition, and onboard or post-processed land classification capability tailored to specific regional or environmental needs.

Therefore, there exists a critical need to develop a custom-built, fixed-wing UAV system optimized for land use image classification—one that is affordable, reliable, and capable of autonomous operation in varied field conditions. This research project addresses that need by designing and implementing a fixed-wing UAV equipped with geo-referenced imaging systems and an image-processing pipeline to enable accurate and efficient land use classification.